\documentclass[a4paper,11pt]{article}

\usepackage{mlsubmit}

\begin{document}

\initmlsubmision{4} % assignment number
{Avijit Roy}   % your name
{18111404}	% your roll number

\begin{mlsolution}

\section{Eigenchangers!}
X is an $N \times D$ matrix over $\mathbb{R}$. An Singular Value Decomposition(SVD) of X is $X = U \Sigma V^T$,where-

1.The columns of U and V are orthonormal.

2. if $\Sigma$ =($\sigma_{ij}$) then $\sigma_{ij}$ = 0 for $i\neq$ 0 and $\sigma_{11} \geq \sigma_{22} \geq ... \geq 0$. Let $\sigma_{ii} = \sigma_i$ are called singular values of X.

\color{blue} Theorem: \color{black}Let X = $U \Sigma V^T$ be an SVD of A. Then,

1. $\sigma_i$'s are the square roots of the eigenvalues of $X^TX$ (As well as $XX^T$).

2. The columns of V are the eigenvectors of $X^TX$. 

3. The columns of U are the eigenvectors of $XX^T$

\color{blue} Proof: 

\color{black}Columns of V are $(v_1,...,v_n)$

Columns of U are $(u_1,...,u_n)$
$$X^TXv_i = (U\Sigma V^T)^TU\Sigma V^Tv_i$$
$$=(V^T)^T \Sigma^T U^TU\Sigma V^T v_i$$
$$=V\Sigma^T \Sigma V^Tv_i =V\Sigma^T\Sigma e_i$$
$$=V\Sigma_i^2e_i = \sigma_i^2v_i$$
Here, We get the value of $\Sigma^T \Sigma$ as-
$$\Sigma^T\Sigma=\begin{bmatrix}
    \sigma_1       & 0 & 0 & \dots & 0 \\
    0       & \sigma_2 & 0 & \dots & 0 \\
    \hdotsfor{5} \\
    0       & 0 & 0 & 0 & \sigma_r \\
        \hdotsfor{5} \\
	0       & 0 & 0 & 0 & 0
\end{bmatrix}
\begin{bmatrix}
    \sigma_1       & 0 & 0 & \dots & 0 \\
    0       & \sigma_2 & 0 & \dots & 0 \\
    \hdotsfor{5} \\
    0       & 0 & 0 & 0 & 0 \\
        \hdotsfor{5} \\
	0       & 0 & 0 & \sigma_r & 0
\end{bmatrix} e_i$$
$$=\begin{bmatrix}
    \sigma_1^2       & 0 & 0 & \dots & 0 \\
    0       & \sigma_2^2 & 0 & \dots & 0 \\
    \hdotsfor{5} \\
    0       & 0 & 0 & \sigma_r^2 & 0 \\
        \hdotsfor{5} \\
	0       & 0 & 0 & 0 & 0
\end{bmatrix}_i = \sigma^2_i \times e_i$$
$\therefore\; v_i$ is an eigenvector of $X^TX$ with eigenvalue $\sigma_i^2$

The diagonal entries of $\Sigma$ are the roots of the eigenvalues of $X^TX $.

Same argument shows that $u_i$'s are the eigenvectors of $X^TX$ with eigenvalues $\sigma_i^2$. 

All the columns of U and V are linearly independent(as they are orthogonal). So, U and V invertible.
$$XV = U\Sigma\;\;\;\;\;\;\;\;\;\;[V^T = V^{-1}]$$
$$=(u_1,...,u_m)\Sigma$$
$$=(\sigma u_1,\sigma u_2,...,\sigma_m u_m)$$
$$\therefore Xv_i = \sigma_i u_i $$
$$\boxed{ u_i=\frac{Xv_i}{\sigma_i}}$$
\color{blue}Advantage: \color{black}
Determining the eigenvalue of $XX^T(N\times N)$ will take $O(N^3)$ and for $X^TX (D \times D)$ it will take $O(D^3)$. As $N \leq D$, the time complexity will be less for $XX^T$
case.
\end{mlsolution}

\begin{mlsolution} 

\section{A General Activation Function}
1. $\beta = kx$ where $k \rightarrow \infty$ will approximate the Identity function.
\begin{flushleft}
2. $\beta \rightarrow \infty$ then h(x) will approximate ReLU.
\end{flushleft}


\end{mlsolution}

\begin{mlsolution}

My solution to problem 3

\end{mlsolution}

\begin{mlsolution}

My solution to problem 4

\end{mlsolution}
	
\begin{mlsolution}

My solution to problem 5

\end{mlsolution}

\begin{mlsolution}

My solution to problem 6

\end{mlsolution}


\end{document}
